\documentclass[11pt]{article}

\title{\textbf{General Knowledge Chatbot: Working document}}
\author{Audrey Bramy, Ruben Dorado}
\date{\today}
\begin{document}

\maketitle


\section{Introduction}


\subsection{Justification}

It is impossible to know everything. Nowadays thanks to the Internet you can ask all your questions to your best search engine and find a considerable amount of results and links to find your answer. The problem is that it is not always easy to find the answer quickly, even if your question is simple.
(For instance, the date of the second war!).

First, when you have a question it is sometimes much faster just to ask a friend. Secondly, you have not always access to the Internet or time to browse the results.

After writing my thesis, I concluded that the SMS exchange could be more intuitive. So I thought about an intelligent chatbot reachable by SMS to ask questions as you can do with a Friend. Moreover, you always have access to SMS with your mobile phone.


\subsection{To do...}

Things to do...

\section{Technical stuff}



The first implementation of the bot is on the file called \texttt{bot1.bot}. It contains two states and a function to process the question.

The bot can be run by typing the following command in a terminal:

\texttt{python src/runbot.py -i bot1.bot}


\bibliographystyle{natbib}
\bibliography{bibtex}

\end{document}